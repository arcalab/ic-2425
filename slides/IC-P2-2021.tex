% !TEX encoding = IsoLatin
\documentclass[titlepage,12pt]{article}
     
\usepackage{amsmath}
\usepackage{amssymb}
\usepackage{stmaryrd}
\usepackage{a4wide}

\def\rtrano#1{\stackrel{#1}{\Longrightarrow}} 
\def\rtran#1{\stackrel{#1}{\longrightarrow}}
\def\tran#1{\stackrel{#1}{\longrightarrow}}
\def\PP{\mathbf{P}}
\def\scong{\equiv}
\def\naobs{\mathbin{\not \approx}} 
\def\fuc#1{#1}
\def\ev{\Diamond}
\def\dimp{\mathbin{\Leftrightarrow}}
\def\imp{\mathbin{\Rightarrow}}
\def\rimp{\mathbin{\Leftarrow}}
\def\limp{\mathbin{\multimap}}
\def\e{\mathbin{\wedge}}
\def\ou{\mathbin{\vee}}
\def\abv{\stackrel{\rm abv}{=}}    
\def\universal#1#2{\forall_{#1}\;.\; #2}
\def\existential#1#2{\exists_{#1}\;.\; #2}
\def\true{\fuc{true}}
\def\false{\fuc{false}}
\def\pv#1#2{\langle #1 \rangle #2}
\def\nc#1#2{[#1]#2}
\def\pvo#1#2{\langle \! \! \! \langle #1 \rangle \! \! \! \rangle\, #2}
\def\nco#1#2{\llbracket #1 \rrbracket #2}
\def\cvg#1{\llbracket \downarrow \rrbracket #1}
\def\cvgr#1#2{\llbracket #1 \downarrow \rrbracket #2}
\def\cvgl#1#2{\llbracket \downarrow  #1 \rrbracket #2}
\def\cvglr#1#2{\llbracket \downarrow  #1 \downarrow \rrbracket #2}
\def\lfp#1#2{\mu {#1}\, .\, {#2}}
\def\lpf#1#2{\mu {#1}\, .\, {#2}}
\def\gfp#1#2{\nu {#1}\, .\, {#2}}
\def\gpf#1#2{\nu {#1}\, .\, {#2}}
\def\mset#1{\vvv #1 \vvv}
\def\vvv{\vert \! \vert}
\def\mnc#1{\vvv [#1] \vvv}
\def\mpv#1{\vvv \langle #1 \rangle \vvv}
\def\bcomp#1{#1^{\text{c}}}
\def\setdef#1#2{\{#1\; |\;#2\}}             % { f(x) | p(x) }
\def\enset#1{\mathopen{ \{ }#1\mathclose{ \} }} % {a,b,...z}
\def\st{. \; }                           % "such that"


\begin{document}
\hline
\begin{center}
\textbf{Interaction and Concurrency}~\\ ~\\
\textbf{Problem Set - 2 } \\
19 April 2021 - 3 May 2021\\
~\\
\end{center}


\hline

\vspace{1.5cm}

\emph{ Please provide a complete, individual answer and quote suitably any reference used.}
\vspace{1cm}

The modal connectives  introduced in the lectures explore the structure of
transitions in $ \PP $, i.e. binary relations $\tran{x} \subseteq \PP \times \PP$, for $ x \in Act$,
 relating the validity of the formulas to sets of
states reached through certain transitions.

It is possible, however,  to define other transition relations in $\PP$
that are computationally relevant.
One of them is \emph {observable transitions} 
$ \rtrano{a} \; \subseteq \; \PP \times \PP $, labelled from  $ L = (Act - \enset{\tau})  \cup \enset{\epsilon}$.
Remember that a $ \rtrano{\epsilon}$-transition corresponds to zero or more transitions through an
unobservable action  $\tau$.

Consider two new modal operators
that express, respectively, the \emph{possibility}
and the \emph{need} for a property to be valid
after performing an arbitrary amount of unobservable behaviour.

\begin{align*}
E  \models \pvo{~}{\phi} & \text{~~~~~~ iff ~~~}
\existential{F \in \setdef{E'}{E \rtrano{\epsilon} E'}}{F \models \phi} \\
E  \models \nco{~}{\phi} & \text{~~~~~~ iff ~~~}
\universal{F \in \setdef{E'}{E \rtrano{\epsilon} E'}}{F \models \phi} 
\end{align*}

By abbreviation we can now define the ``observable versions '' of
$\pv{K}{~}$ e $\nc{K}{~}$, for $K \subseteq L$.  Thus,
\begin{align*}
\pvo{K}{\phi} \abv\; &  \pvo{~}{~} \pv{K}{~}\pvo{~}{\phi} \\
\nco{K}{\phi} \abv\; &  \nco{~}{~} \nc{K}{~}\nco{~}{\phi} 
\end{align*}

Another relevant operator is defined by 

\begin{equation*}
E\:  \models\: \cvg{\phi} \text{~~~~~~ iff ~~~} E \downarrow\: \e\:
\universal{F \in \setdef{E'}{E \rtrano{\epsilon} E'}}{F \models \phi}
\end{equation*}

\noindent
where $E \downarrow$ means that process $E$ is \emph{convergent}, i.e. it does not commit to an infinite loop of 
internal actions ($\tau$).


\begin{enumerate}
\item
Explain in your own words the meaning of the three new logical operators just introduced. For each of them specify a property resorting to it and describes in your own words its intended meaning.
\item
Explain the meaning of formulas $\pvo{abac}{\true}$ and $\nco{-}{\false}$. Illustrate their use through the specification of four different, non-bisimilar processes such that $\pvo{abac}{\true}$ holds in two of them and $\nco{-}{\false}$ in the other two.
\item
In the logic you have studied in the lectures, formula
$$
\pv{-}{\true} \e \nc{-a}{\false}
$$
expresses \emph{inevitability}, i.e. the occurrence of action $a$ is inevitable.
Which of the  formulas
\begin{enumerate}
\item  $\pvo{-}{\true} \e \nco{-a}{\false}$  
\item $\nco{~}{~} \pvo{-}{\true} \e \nco{-a}{\false}$
\end{enumerate}
if any, would express a similar property in the observational setting? 
Justify your answer. If none seems suitable, provide an alternative specification.

\item
In the lectures, you have studied a close relationship between \emph{bisimilarity} and \emph{modal equivalence} for the logic then introduced.
Discuss in some detail if and how a similar result holds relating \emph{observational equivalence} and \emph{modal equivalence} 
for the extended logic.
\end{enumerate}

\end{document}

